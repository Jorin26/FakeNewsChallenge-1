\documentclass[12pt]{article}
\usepackage[margin=1in]{geometry}
\usepackage{hyperref}


\title{Stance Detection for the Fake News Challenge Dataset using Deep Learning}

\author{Anak Agung Ngurah Bagus Trihatmaja and Shishir Kurhade}

\date{}

\begin{document}

\maketitle

\abstract
The problem of fake news has arisen recently as a threat to high-quality 
journalism and well-informed public disclosure. The goal of fake news 
challenge is to explore how artificial intelligence technologies, particularly
machine learning and natural language processing, might be leveraged to combat 
the fake news problem \cite{fake_news_challenge}. The goal of our project is to develop machine learning 
models to predict a stance label (‘Agrees’, ’Disagrees’, ’Related’, ’Unrelated’) 
with respect to the title for a respective news article. For this purpose, we 
will use the gated recurrent unit (GRU) to predict the labels. As a baseline to 
measure performance we will also solve the problem using logistic regression method.


\section{Introduction}
In this project, we try to combat a serious problem in our media using machine 
learning techniques. In a poll conducted by Pew Research Center, 64\% of US 
adults said that fake news has caused a “great deal of confusion” about the 
basic facts of current issues and events \cite{barthel_mitchell_holcomb_2016}. 
This problem views the task of fake-news detection as a stance detection problem 
which is a labeling task. We want to automatically classify a news into four labels, 
which are ‘unrelated’, ‘agrees’, ‘disagrees’, and ‘discusses’.  

A reasoning for these labels is as follows:
\begin{enumerate}
  \item \textbf{Agrees}: The body text agrees with the headline.
  \item \textbf{Disagrees}: The body text disagrees with the headline.
  \item \textbf{Discusses}: The body text discuss the same topic as the headline, 
    but does not take a position
  \item \textbf{Unrelated}: The body text discusses a different topic than the headline
\end{enumerate}

The classifier that we build could later be used as a base of a 
fake news detection tool that can automatically categorize the news into the 
stances given.

\section{Proposed Project}
\subsection{Dataset Overview}

The data provided by the Fake News Challenge consists of 
headline, body, and stace. 
For training, there are two csv files:
\begin{enumerate}
  \item \textit{train\_bodies.csv}: contains the body text of articles with its ID
  \item \textit{train\_stances.csv}: contains labeled stances for pairs of article 
      headlines and article bodies, in which the article bodies refer to the 
      bodies in train\_bodies.csv
\end{enumerate}

The distribution of the data is as follows:
\begin{center}
  \begin{tabular} 
    {|c|c|c|c|c|}
    \hline
    Rows & Unrelated & Discuss & Agree & Disagree \\
    \hline
    49972 & 0.73131 & 0.17828 & 0.0736012 & 0.0168094 \\
    \hline
  \end{tabular}
\end{center}
We will roughly use 4000 samples as our development set, for choosing 
hyperparameter and performance evaluation, and use the rest for 
our training data. 

\subsection{Methods}
\subsubsection{Pre-processing}

The text from the corpus will be converted to tokens using \textit{‘nltk’} package and 
then be mapped to corresponding vectorized forms using  pre-trained \textit{GloVe} 
representations freely available on the Stanford NLP group website \cite{Bird:2009:NLP:1717171,pennington2014glove}. 
Since the text sequences observed will be of variable length we will pad all 
sequences to the length of the maximum length text sequence before inputting 
it to our model. The data pre-processing includes normalizing the case, handling the punctuation
and non-alphabetic symbols.

\subsubsection{Learning Model}

We will use a learning model that is  RNN variant to predict the stances. 
The reason why it is based on RNN instead of CNN is that, for language 
modeling, RNN models are still the best approach \cite{MikolovDKBC11}. 
The difficulty with RNN models is that they are hard to train because they 
suffer from the vanishing gradient problem \cite{Murphy:2012:MLP:2380985}. 
The solution to this problem is to use RNN variant called long short-term memory
(LSTM). However, recently a model called gate recurrent unit (GRU) was 
introduced by Cho et al. in 2014 \cite{DBLP:journals/corr/ChoMGBSB14}. GRU is similar with LSTM. Unlike LSTM, GRU 
combines forget and input gates into a single “update gate” and merges the cell
state and hidden state. This makes GRU computationally more efficient than 
LSTM and GRU model has been increasingly popular \cite{DBLP:journals/corr/Rana16a}. For this project, we will use 
TensorFlow API for GRU, \textit{tf.nn.rnn\_cell.GRUCell} \cite{tensorflow2015-whitepaper}. 

It is often said, that neural network performs better for a case like this.
To compare the performance of GRU RNNs we will also perform logistic regression
to generate stances as a baseline for our project by building it from scratch.

\subsubsection{Scoring}
To measure the accuracy, we will use the evaluation tool provided by 
Fake News Challenge. The tool will evaluate our model and output a score. 
The score is weighted as follows:
\begin{enumerate}
    \item Classifying pair of body and headline as related and unrelated is 
      weighted 25\%
    \item Classifying related pairs as agrees, disagrees, or discusses is 
      weigted as 75\%
\end{enumerate}
The related/unrelated classification task is expected to be much easier while classifying the agrees, disagrees or discuss 
is more difficult and more relevant to fake news detection.

\bibliographystyle{unsrt}
\bibliography{2380985}

\end{document}
